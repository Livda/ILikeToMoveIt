
		\section{La méthode actuellement en place}
	

La gestion de la mobilité sortante n'utilise actuellement pas de logiciel dédié . 
Pour combler ce manque, les \ris combinent l'utilisation de plusieurs outils et techniques différents, ce qui peut être compliqué et engendre des pertes de temps. 
Nous présenterons ainsi ce système, puis aborderons un autre logiciel qui pourrait être utilisé pour gérer les mobilités.

\medbreak

Le comportement de l'application souhaité, décrit en section \ref{sec::gestion_mobilite} est grandement inspiré des démarches actuelles. 
Ainsi, les trois grandes étapes décrites plus haut sont également présentes. 
Notons que le fonctionnement  décrit est celui utilisé au département informatique.  
 \subsection{Le choix des destinations de départ et leur affectation}
		 
		Pour ce faire, les \ris créent un formulaire en ligne demandant les différents vœux, puis l'envoient par mail aux étudiants. 
		
		Une fois reçu, les étudiants le remplissent. À la date limite, le \ri ferme le sondage, et récupère un fichier excel qui récapitule les différentes requêtes. 
		
		Christian Raymond a créé un script \textit{perl} qui lui permet alors de classer les étudiants grâce au fichier récupéré.
		
		Par la suite, les résultats  précisant les affectations sont envoyés par mail. 
		
		\medbreak
		
		Les principales limites de cette partie sont les suivantes : 
		\begin{itemize}
		\item le choix par formulaire en ligne est relativement peu sécurisé, et peu ergonomique. Un étudiant peut  éventuellement s'inscrire plusieurs fois.
		
		Nous pouvons également citer la difficulté de gestion du choix entre mobilité classique et double diplôme. Un étudiant doit pouvoir choisir : soit il effectue une mobilité d'un semestre, soit un double diplôme. Dans le cas actuel, l'étudiant à la possibilité d'émettre des vœux pour les deux types de programmes ;
		
		\item la partie affectation est fastidieuse  pour les \ris. En effet, il faut faire un nombre important de manipulations pour obtenir le résultat  souhaité. De plus, ce système n'est pas flexible. Si le format de retour du formulaire est modifié, il faut adapter le script ;

		\item certaines étapes sont redondantes. Le formulaire créé est toujours le même. Cependant il faut ré-effectuer l'opération chaque année. Il s'agit d'une perte de temps.
			\end{itemize}

		\subsection{La gestion  des documents}
		La gestion des documents administratif est entièrement fait à la main. Les étudiants téléchargent les documents sur la plateforme moodle, ou sont envoyés par mail pour les cas particuliers. 
                
                Par la suite, les étudiants rencontre le \ri pour faire signer les documents. Une date butoir est fixée par le SRI pour le rendu de ce document, afin de pouvoir répartir les subventions en temps et en heure. 

\medbreak

Côté étudiant, compléter le \textit{Learning Agreement} demande à remplir un certain nombre de champs qui sont déjà connus par l'administration. On peut citer le nom de l'étudiant, sa filière, mais aussi quelques informations concernant l'INSA, qui peuvent être compliqué à trouver. 

\medbreak

L'idée dans cette partie est donc de simplifier au maximum la tache pour les étudiants. Tout d'abord en recensant les documents à produire sur l'application, ensuite en offrant une interface qui permettra de remplir certains documents plus facilement. 

\subsection{Le suivi de la mobilité}

Dans la majorité des cas, le contrat d'étude est modifié une fois arrivé sur place. Le premier contrat, rendu bien plus tôt, est invalide. Il faut donc procéder à une nouvelle signature du contrat. 

Par conséquent, le contrat est rédigé dans l'université d'accueil, par la suite  signé par l'administration de cet établissement. Alors, l'étudiant scanne  le document puis l'envoie par mail au \ri. 

Celui ci l'imprime alors, le signe, le scanne puis le renvoie à l'étudiant. Le contrat est  de nouveau imprimé, et remis à l'université d'accueil. 

\medbreak

Les limites de ce fonctionnement sont évidentes : non seulement le processus est fastidieux, mais en plus peu écologique. L'idée décrite plus haut serait donc de ne pas avoir besoin d'imprimer le document pour le \ri, évitant ainsi une perte de temps et réduisant le gaspillage de papier. 

\bigbreak 

Ainsi il est facile de constater que le fonctionnement actuel n'est pas le plus optimisé. Les pertes de temps pour les \ris et l'ergonomie de la méthode mettent en évidence le besoin d'un outil dédié à la gestion de mobilité. 

Une solution de ce type existe. Il s'agit du logiciel \textit{MoveON 4}, développé par la société \textsc{QS unisolution}. Il permet d'implémenter la quasi-totalité des besoins du cahier des charges. Le doute subsiste sur le classement automatique des élèves, selon les critères abordés plus haut. Il est probable que l'option existe, cependant rien de tel n'est précisé sur le site web de l'entreprise. 

De même, nous ne savons pas si le logiciel permet de séparer les étudiants en départements, ou encore s'il est possible de générer nativement la création de fiches de jury. 

Autre inconvénient, le logiciel est payant. Cependant, l'INSA est déjà utilisatrice de la version 3 du logiciel. 
Nous ne savons pas précisément quelles y sont les fonctionnalités présentes. Nous avons cependant demandé l'autorisation au SRI de tester le logiciel, demande qui a été accueillie favorablement. Nous prévoyons également de tester la version 4 du logiciel, puisque l'INSA à pour projet de l'acheter en Juin 2016. Notre demande est cependant encore en cours de soumission. 


Malgré ces doutes, il semblerait possible d'utiliser ce logiciel pour satisfaire la plupart des exigences du cahier des charges. 
