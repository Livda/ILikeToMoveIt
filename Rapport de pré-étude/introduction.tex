\chapter*{Introduction}
\addcontentsline{toc}{chapter}{Introduction}	
Dans le cadre d'une ouverture avec le monde extérieur, l'INSA demande à ce que chaque élève effectue un séjour à l'étranger sous forme de mobilité international, qui peut se faire sous plusieurs formes:
\begin{itemize}
\item semestre d'étude;
\item double diplôme;
\item stage.
\end{itemize}

La gestion du départ de tous les étudiants est actuellement une tâche plutôt laborieuse.
Ces départs s'organisent via plusieurs étapes (choix de destination, validation des choix, choix des cours, mise ligne des relevés de notes). Il est donc préférable d'automatiser et de faciliter un maximum de manipulations.

Notre projet intervient à cette étape, dans le but d'améliorer les méthodes, plus ou moins efficaces, utilisées pour le moment. Nous devons ainsi créer une application web qui permettrait dans un premier temps de faciliter la tache du correspondant RI (Relations Internationales) du département Informatique. L'objectif, dans un second temps, est d'étendre ce site pour tous les départements de l'INSA.

Nous allons pour cela reprendre un prototype réalisé l'année dernière dans le cadre de l'étude pratique en 3INFO, et l'améliorer pour le rendre fonctionnel.
Nous allons essayer de rendre une version opérationnelle avant la campagne  de décembre pour pouvoir tester notre site dans des conditions réelles.
