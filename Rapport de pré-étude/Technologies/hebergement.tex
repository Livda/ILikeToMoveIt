\section*{•} du serveur}

Pour ce qui est de l’hébergement de notre application, nous nous sommes naturellement tournés vers le CRI, qui propose des VPS pour les projets étudiants. Au terme d’une réunion avec le personnel, ils nous ont offert l’accès à un de ces serveurs auquel nous avons choisi le nom de domaine getawayfrom.insa-rennes.fr. L’avantage de cette solution est que nous aurions à disposition un serveur où nous serions libres de développer comme nous le souhaitons. Le problème est que la connexion se fait en SSH, uniquement via le réseau de l’INSA. De plus, ce VPS est censé être libéré à la fin de l’année, alors que notre application doit rester pérenne, voire être intégrée à l’ENT avec acceptation du CRI. Nous devrons donc discuter d’un hébergement permanent ultérieurement.

En outre, nous souhaitons utiliser CAS de l’INSA pour que les étudiants disposent d’un accès sécurisé à leurs choix de destinations.

Enfin, nous voulions disposer de certaines informations sur les étudiant afin d’alimenter notre base de données. Le CRI a accepté de nous donner l’accès à une vue sur le LDAP de l’INSA. Cet accès ne sera donné qu’après validation par la CNIL d’une demande formulée par le CRI, relative à la sécurité des données confidentielles.