\section{Hébergement du serveur}

Pour ce qui est de l'hébergement de notre application, nous nous sommes naturellement tournés vers le CRI, qui propose des VPS (Serveur Virtuel Privé) pour les projets étudiants. Au terme d'une réunion avec le personnel, ils nous ont offert l'accès à un serveur auquel nous avons choisi le nom de domaine getawayfrom.insa-rennes.fr. L'avantage de cette solution est que nous avons à disposition un serveur où nous sommes libres de développer comme nous le souhaitons. Le problème est que la connexion se fait en SSH, uniquement via le réseau de l'INSA. De plus, un VPS comme ceux-là, proposé par le CRI, est censé être libéré à la fin de l'année, alors que notre application doit rester pérenne, voire être intégrée à l'ENT avec acceptation du CRI. Nous avons donc discuté avec certains membres du CRI afin de mettre en place l'application pour qu'eux la maintienne dans les meilleures conditions.

En outre, nous souhaitons utiliser le CAS (Central Authentification System) de l'INSA pour que les étudiants puissent, après authentification, disposer d'un accès sécurisé à leurs choix de destinations.

Enfin, nous voulions avoir accès à certaines informations sur les étudiant afin d'alimenter notre base de données. Le CRI a accepté de nous donner l'accès à une vue sur le LDAP de l'INSA. Cet accès aurait pu n'être donné qu'après validation par la CNIL d'une demande formulée par le CRI, relative à la sécurité des données confidentielles. Cependant, dans notre cas, ces informations ne sont pas sensibles au niveau protection de la vie privée, et sont donc accessibles sans aval de la CNIL.