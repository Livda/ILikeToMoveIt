\section{La gestion de la base de donnée}

L'utilisation du framework \symfony nous permet en grande partie d'abstraire la question de la base de donnée : il n'y aura aucune requête en \sql dans notre code. 
Le choix du système de gestion de base de donnée (SGBD) n'est cependant pas  à négliger, puisque les performances du site web peuvent être grandement affectées par un mauvais choix de système de gestion. 

Le prototype déjà existant utilisait \mysql. 
De plus, lors de notre réunion avec les agents du Centre des Ressources en Informatique (CRI), il nous a été conseillé de choisir entre ce système et \psql. 
Le choix le plus simple serait donc de continuer avec ce qui a déjà été utilisé. 
Cependant, le changement du système est extrêmement simple à effectuer, nous allons donc légitimement  discuter des avantages et inconvénients de ces deux logiciels, mais également de certains autres, extrêmement présents dans ce domaine. 

\subsection{SqLite ou Oracle, deux solutions inappropriées}

Nous commencerons premièrement par deux solutions qui ne sont pas adaptées à notre problème : Oracle Database et SqLite. 

\paragraph*{Oracle Database } est un \textit{SGBD} créé par \textit{Larry Ellison}, et racheté depuis par Oracle Corporation. C'est aujourd'hui le système le plus utilisé dans le monde de l'entreprise, grâce aux fonctionnalités qu'il propose. 
Cependant, les fonctionnalités supplémentaires sont utiles pour un nombre élevé d'utilisateurs en même temps, ou pour un grand nombre de données à stocker. Notre application ne connaitra jamais plus de 500 utilisateurs simultanées, et notre serveur sera limité en taille à 30 gigaoctets. Les plus d'Oracle ne nous seront donc pas utiles. 

L'inconvénient majeur d'Oracle est le prix. Le logiciel est relativement cher, d'autant plus pour des propriétés inutiles dans notre cas.
Nous ne nous tournerons donc pas vers cette solution, ni vers toute solutions payantes. Les licenses des SGBD donc nous parlerons maintenant seront donc toutes gratuites.  


\paragraph*{SqLite} est une bibliothèque légère, ce qui lui permet d'être facilement intégrée aux applications. C'est la bibliothèque la plus distribuée au monde, par example Firefox, iTunes ou Androïd l'utilisent pour gérer leurs données. 


S'il est possible de l'utiliser pour un site web, le logiciel n'a pas été prévu pour cet usage ; il est censé gérer quelques utilisateurs, et un faible jeu  de données. L'utilisation simultanée par un nombre plus élevé d'utilisateurs ralentirait considérablement le fonctionnement de l'application. Nous ne retiendrons donc pas cette solution, ou  tout autre système embarqué. 

\subsection{MySQL et MariaDB}
 
 Comme indiqué en introduction, \mysql est le SGBD utilisé lors de la création du prototype. Ce système, possédé par Oracle Corporation, est sous double licence libre et propriétaire. LLa license de \mysql peut donc être payante ou gratuite selon la nature du produit : propriétaire ou libre. Dans notre cas, nous utliserons l'API tierce \symfony, nous n'utiliserons donc pas directement \mysql. Ici, sa license est donc gratuite. 
 
 \mysql est un des SQBG les plus utilisé. Ainsi des sites web comme \textit{Google}, \textit{Yahoo}, ou encore \textit{Airbus}	l'utilisent. En effet, \mysql fait parti du quatuor \textit{LAMP} (Linux Apache MySQL PHP) fréquemment utilisé par les développeurs web, principalement en raison de son faible coût. 
 
 \smallbreak
 
 Les principaux avantages de \mysql sont les suivants : 
 \begin{itemize}
 \item peu de modification à faire lorsque l'on passe le projet en production  ;
 \item certaines règles ne sont pas présentes dans d'autre SGBD (comme les vues) ;
 \item efficace avec les faibles jeux de données. En effet, \mysql ne fait pas d'\textit{intégrité référentielle}, ce qui lui permet de gagner en rapidité, au dépend de l'intégrité de la base de donnée. Ainsi, si le site est utilisé par un nombre important d'utilisateurs, le risque d'erreur d'intégrité augmente. 
 \end{itemize}

Voyons maintenant les inconvénients :

\begin{itemize}
\item documentation assez obscure ;
\item prend quelques écarts avec la norme SQL2003 ;
\item moins efficace avec les gros jeux de données. 
\end{itemize} 

Citons une dernière critique majeure de \mysql, apparue lors du rachat du logiciel par Oracle. Le système est alors  passé de libre à propriétaire, ce qui  a fait naitre de nombreuses craintes par la communauté du libre, auquel s'ajoute également un de ses créateurs Michael Widenius. 

Il faut savoir que la société Oracle fait généralement peur aux libristes. L'entreprise avait déjà acheté un certain nombre de projets libres. C'est le cas du projet Open Office, qui n'a pas connu de mise à jour majeure depuis son rachat. 

La crainte que le projet soit arrêté a incité Widenius à créer un fork, copie libre du logiciel sur laquelle travaille la communauté du libre. Le projet du nom de \mdb, accepte toutes les fonctionnalités  de \mysql. 

Actuellement, Oracle produit des mises à jours sans indiquer leur contenus. On n'en connait donc pas l'utilité réelle, et surtout cela ne permet pas de savoir quelles failles ont été corrigées. \mdb au contraire assure que les failles soient corrigées. L'utilisation de ce système assure donc une certaine sureté au site web. 

De plus, de nombreuses fonctionnalités ont été ajoutées au système. Ainsi, \mdb assure avoir de meilleurs performances que \mysql, notamment lorsqu'elle est utilisée pour de la big data. Le principal inconvénient dont nous parlions en devient caduc. 

\medbreak

Ainsi, entre \mysql et \mdb, nous choisirons d'utiliser \mdb, pour ses meilleurs performances et sa plus grande sécurité. De plus, si le projet est amené à durer plusieurs années, le jeu de données, qui augmentera en taille, ne sera pas problématique. Cependant, nous souhaitons quand même comparer \mdb avec \psql, qui semble également prometteur. 


\subsection{PostgreSQL}

\psql est un SGBD très apprécié des développeurs. C'était d'ailleurs le système qui nous avait été enseigné en deuxième année, lors du module d'initiation aux bases de données. 

Contrairement à \mysql, le projet est libre. Par conséquent, sa license est donc gratuite\footnote{sous licence BSD}. De plus, la communauté qui gère ce projet est très active, ce qui permet un développement réactif et une correction rapide aux différentes failles de sécurité qui peuvent être découverte. 
Enfin, chaque version est entièrement testée avant sa sortie. \psql est ainsi un SGBD sûr. 

Voyons maintenant les avantages de \psql par rapport à \mdb : 

\begin{itemize}
\item c'est le SGBD qui respecte le plus la norme SQL2003. Il est ainsi plus universel que \mdb ;
\item une documentation très claire (ce qui n'est pas à prendre en compte dans notre cas, puisque \symfony abstrait l'utilisation direct du SGBD) ;
\item extrêmement stable, comme précisé plus haut ;
\item extrêmement efficace avec les gros jeux de données. Ainsi, \textit{Météo France} l'utilise pour gérer plus de 4 To de données. Notons que cet avantage n'est pas à prendre en compte non plus : nous ne gérerons jamais plus de 30 Go ;
\item la cohérence de la base est garantie.
\end{itemize}

\psql a néanmoins quelques inconvénients majeurs  par rapport à \mdb : 

\begin{itemize}
\item il est inutilisable tel quel en production. Il faudra donc attentivement configurer le système ;
\item il nécessite une maintenance en production pour que la base reste efficace ;
\item il est moins efficace avec les très petites bases de données.  
\end{itemize}

\bigbreak
Ainsi, il semble que \psql ne soit pas le SGBD le plus adapté à notre projet, mais plus pour les projets demandant de plus grosses bases de données. 
Les principaux avantages de ce système ne nous seront pas utiles. De plus, l'utilisation de \psql nous rajouterais une dose de travail. Comme le projet est conséquent, et limité dans le temps, il n'est pas forcément souhaitable de rajouter une charge de travail pour un gain pas forcément utile. 

Au contraire \mdb est à priori plus adapté. En effet, il sera très simple à configurer, et gère mieux les bases de données de la taille de celle qu'on attend. Si celle si vient à grossir, le système est plus optimisé que \mysql pour les données plus grosses. Nous utiliserons donc \mdb. 
