\section{La gestion de la base de donnée}

L'utilisation du framework \symfony nous permet en grande partie d'abstraire la question de la base de donnée : il n'y aura aucune requête en \sql dans notre code. 
Le choix du système de gestion de base de donnée n'est cependant pas  à négliger, puisque les performances du site web peuvent être grandement affectées par un mauvais choix de système de gestion. 

Le prototype déjà existant utilisait \mysql. 
De plus, lors de notre réunion avec les agents du \textit{Centre des Ressources en Informatique} (CRI), il nous a été conseillé de choisir entre ce système et \psql. 
Le choix le plus simple serait donc de continuer avec ce qui a déjà été utilisé. 
Cependant, le changement du système est extrêmement simple à effectuer, nous allons donc légitimement  discuter des avantages et inconvénients de ces deux logiciels, mais également de certains autres, extrêmement présent dans ce domaine. 

% ceci est un brouillon
%- mysql
%- mariadb
%- postgresql

%mysql  est une solution open source, rachetée par Oracle. Fork par Widenius qui s'appelle  
