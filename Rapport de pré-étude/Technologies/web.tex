\section{Côté serveur}
Le choix technologique central de notre projet est le choix du langage côté serveur. De nombreux choix s'offraient à nous, notamment: Ruby, NodeJs, Java EE, PhP, Perl, etc
Pour des questions de rapidité de développement, un prototype fonctionnel nous étant demander pour fin décembre. Nous avons restreint nos choix à deux technologie: Java EE, pour sa prise en main facile et sa maitrise par certains membres de l'équipe et PhP pour sa rapidité d'apprentissage, mais essentiellement car comme vous l'avez vue dans la partie 2, nous possédons un début de prototype développer en Php sous le framework Symfony2.

piti tableau:
Java ee 
                       
Rapidité d'apprentissage:     ++ un seul langage à maitrisé  
 
Gestion de la base de donnée:  - toutes les classes de lecture et d'écriture à développer, et création des classes pour le pattern DAO (Data Access Object) 

Rapidité de développement: ++Rapide, mais nous repartons de zéro


Php avec Symfony2
  
Rapidité d'apprentissage:   + un langage et une méthode de développement pour respecter les normes symfony2

Gestion de la base de donnée:  ++ tout est géré automatiquement par Doctrine, une fonctionnalitée de Symfony2, quelques méthodes d'accès sont à redéfinir mais c'est minime

Rapidité de développement: +++ Rapide, et nous possédons de bonnes bases

De plus, pour implémenter toute les fonctionnalité dont l'on a besoin (remplissage de pdf, communication avec le CAS et le lpad), Symfony possède une grande communauté ce qui laisse envisagé que des outils résolvant ces problématiques ont déjà été développés.
Le derniers argument qui a tranché est le fait que la norme strict de Symfony2 permet d'avoir une organisation et une reprise du travail d'autruis plus facilement.

Nous développeront donc sous PhP avec le frameworks Symfony2