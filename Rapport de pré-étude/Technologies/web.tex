\section{Côté serveur}

\subsection{Langage serveur}
Le choix technologique central de notre projet est le choix du langage côté serveur. De nombreux choix s'offraient à nous, notamment: Ruby, \textit{NodeJs}, \textit{Java EE}, PHP, \textit{Perl}, etc.

Pour des questions de rapidité de développement, un prototype fonctionnel nous étant demandé pour fin décembre, nous avons restreint nos choix à deux technologies: \textit{Java EE}, pour sa prise en main facile et sa maîtrise par certains membres de l'équipe et PHP pour sa rapidité d'apprentissage, mais essentiellement car comme mentionné dans la partie 2, nous possédons un prototype de notre application développé en PHP sous le framework \textit{Symfony2}. Ce qui permettra de grandement améliorer notre vitesse de développement.\\


\subsubsection{Rapidité d'apprentissage}
\paragraph{Java EE}
L'avantage de \textit{Java EE} c'est qu'il ne faut connaître que le langage Java et apprendre à se servir de quelques bibliothèques dédiées au langage serveur.
Ce langage étant maîtriser déjà par toute l'équipe, l'apprentissage devrais être très rapide et donc le développement vite commencé.

\paragraph{Symfony2}
\textit{Symfony2}nécessite la connaissance du PHP et de toute une norme assez lourde. 
Le PHP et la norme n'étant pas connu par toute l'équipe le temps de formation sera plus long qu'avec \textit{Java EE}.

\subsubsection{Gestion de la base de donnée}
\paragraph{Java EE}
Nous devrons développer toutes les classes d'écriture et de lecture dans la base de donnée. De plus nous devrons crée tout les objets Java destinée à recevoir les données lues dans la base pour suivre le partern DAO et pouvoir se servir des pages dynamiques de \textit{Java EE}.

\paragraph{Symfony2}
\textit{Symfony2}possède un outils appelé Doctrine qui génère automatiquement la base de donnée, les méthodes d'accès et les variables pour permette le traitement dans la page dynamique. Très peu de seront donc développé, notamment les plus poussées.

\subsubsection{Rapidité de développement}
\paragraph{Java EE}
Le développement en \textit{Java EE}est relativement rapide, mais nous devons abandonner le prototype et passer plus de temps dans la création de la base de donnée.

\paragraph{Symfony2}
Avec \textit{Symfony2}nous partons d'un prototype déjà bien avancé et une fois la norme apprise le développement est tout aussi performant que sur \textit{Java EE}.


\begin{tabular}{|m{125pt}|m{150pt}|m{150pt}|}
	\hline
	\null & \textbf{Java EE} & \textbf{Symfony2} \\
	\hline
	Rapidité d'apprentissage & ++ & +-\\
	\hline
	Gestion de la base de données & -- & ++\\
	\hline
	Rapidité de développement & +- & ++ \\
	\hline
\end{tabular} \vspace*{5mm}

Nous pouvons donc conclure que malgré un temps d'apprentissage plus long, une fois celui-ci acquis, l'outils Doctrine, sa norme et prototype nous permettront d'aller plus vite que de tout reprendre à zéro avec \textit{Java EE}.

Pour implémenter toutes les fonctionnalités nécessaire au fonctionnement de l'application (gestion des pdf, communication avec le CAS et le LDAP), \textit{Symfony2} possède une grande communauté, ce qui laisse envisager que des outils résolvant ces problématiques ont déjà été développés.
De plus, le fait que la norme stricte de \textit{Symfony2} permet d'avoir une organisation et une reprise du travail d'autrui plus facilement. Ce qui permettra à notre code d'être maintenu dans le temps.\\

Nous avons donc décidé de développer sous PHP avec le framework \textit{Symfony2}.

\subsection{Technologie serveur}

Nginx VS Apache incoming