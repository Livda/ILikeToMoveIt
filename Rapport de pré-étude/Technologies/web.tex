\section{Côté serveur}
Le choix technologique central de notre projet est le choix du langage côté serveur. De nombreux choix s'offraient à nous, notamment: Ruby, \textit{NodeJs}, \textit{Java EE}, PHP, \textit{Perl}, etc.

Pour des questions de rapidité de développement, un prototype fonctionnel nous étant demandé pour fin décembre, nous avons restreint nos choix à deux technologies: \textit{Java EE}, pour sa prise en main facile et sa maîtrise par certains membres de l'équipe et PHP pour sa rapidité d'apprentissage, mais essentiellement car comme mentionné dans la partie 2, nous possédons un prototype de notre application développé en PHP sous le framework \textit{Symfony2}. Ce qui permettra de grandement améliorer notre vitesse de développement.\\


\begin{tabular}{|m{125pt}|m{150pt}|m{150pt}|}
	\hline
	\null & \textbf{Java EE} & \textbf{Symfony2} \\
	\hline
	Rapidité d'apprentissage & Un  seul langage à maîtriser & Un seul langage et une norme pour simplifier le développement\\
	\hline
	Gestion de la base de données & Toutes les classes de lecture et d'écriture sont à développer. De même pour les classes pour le pattern DAO (Data Acces Object) & Tout est géré automatiquement par Doctrine, une fonctionnalité de Symfony2. Il y a quelques méthodes d'accès à redéfinir, mais le travail est minime\\
	\hline
	Rapidité de développement & Rapide mais nous repartons de zéro & Rapide, et nous possédons un prototype déjà bien avancé\\
	\hline
\end{tabular} \vspace*{5mm}
%Java ee 
%                       
%Rapidité d'apprentissage:     ++ un seul langage à maîtriser
% 
%Gestion de la base de donnée:  - toutes les classes de lecture et d'écriture à développer, et création des classes pour le pattern DAO (Data Access Object)
%
%Rapidité de développement: ++Rapide, mais nous repartons de zéro
%
%
%PPH avec Symfony2
%  
%Rapidité d'apprentissage:   + un langage et une méthode de développement pour respecter les normes Symfony2
%
%Gestion de la base de données:  ++ tout est géré automatiquement par Doctrine, une fonctionnalité de Symfony2, quelques méthodes d'accès sont à redéfinir mais c'est minime
%
%Rapidité de développement: +++ Rapide, et nous possédons de bonnes bases

De plus, pour implémenter toutes les fonctionnalités nécessaire au fonctionnement de l'application (gestion des pdf, communication avec le CAS et le LDAP), \textit{Symfony2} possède une grande communauté, ce qui laisse envisager que des outils résolvant ces problématiques ont déjà été développés.

Le dernier argument qui a tranché est le fait que la norme stricte de \textit{Symfony2} permet d'avoir une organisation et une reprise du travail d'autrui plus facilement. Ce qui permettra à notre code d'être maintenu dans le temps.\\

Nous avons donc décidé de développer sous PHP avec le framework \textit{Symfony2}.