
	\chapter*{Conclusion}
\addcontentsline{toc}{chapter}{Conclusion}	

L'application a pour but final de faciliter le travail des \ris, en effectuant des tâches longues et répétitives, et en centralisant les données et les échanges. Elle pourra ainsi stocker les écoles partenaires, les étudiants et leur statut, les \voe et les affectations principalement. De plus, elle permettra de tenir un historique des départs en mobilité, et d'obtenir des statistiques sur les choix de destinations des étudiants. \\

Les anciennes méthodes étant un peu laborieuses, un prototype a déjà été développé afin d'aider les \ris dans leur tâche. Ce dernier n'est pas totalement implémenté et certaines fonctionnalités ne correspondent pas à nos spécification. Nous avons cependant décidé de l'utiliser afin de gagner en temps de développement.

Le projet va donc se concentrer sur du développement Web en utilisant le framework \symfony, basé sur le langage \php.
Le prototype déjà commencé va nous permettre d'être plus rapide pour le développement, puisque de nombreuses fonctionnalités ont déjà été implémentées. 

Les technologies Web devront aussi être maitrisées : HTML, CSS, ainsi que les bases de données. Nous avons décidé d'utiliser le SGBD \mdb, qui n'est pas sous licence et adapté à la taille des données de notre projet. Concernant l'hébergement, le CRI nous a proposé un VPS, avec des accès au CAS et au LDAP de l'INSA, permettant l'authentification facile des étudiants. Le langage utilisé étant \php, plusieurs choix de serveurs se sont présentés à nous. Nous avons finalement décidé d'utiliser !!!!!!!!!!!!!!!!!!!!!!!!!!!!!!!!!!!!!!!!!!

estimation


Pour aider les \ris lors des échanges de documents, l'idée de les signer électroniquement fut suggérée. Ainsi, certains documents PDF devront pouvoir être générés et une signature apposée. \\

La prochaine étape est donc la rédaction des spécifications du projet, puis la modélisation. Le prototype devra également être modifié pour correspondre à nos besoins. Enfin, l'apprentissage complet des technologies utilisée est primordiale pour l'avancée du projet. Les technologies du Web, ainsi que le framework \symfony et ses normes devront être maitrisés pour développer efficacement.