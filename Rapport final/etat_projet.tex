\chapter{Avancement du projet}

La version finale de notre projet présente aujourd'hui une application fonctionnelle proposant les fonctionnalités nécessaires pour gérer automatiquement les affectations, les learning agreements et le suivit des résultats scolaires des étudiants en mobilité.
Nous avons donc réalisé un des principaux objectif, à savoir proposer un application pouvant servir dans le futur.
\smallbreak
Si le projet devait se poursuivre et être amélioré, voilà quelques fonctionnalités auxquelles il serait bon de s'intéresser :
\begin{itemize}
	\item Nous n'avons pas implémenté l'ajout de commentaires détaillés sur les universités. L'avis des élèves que nous avons interrogés serait de créer une page comprenant toutes les informations utiles pour chaque destination. Il s'agirait donc de créer une sorte de \textit{Wiki} pour les universités partenaires de l'INSA de Rennes.
	\item Il serait intéressant d'intégrer le service RI, en cherchant à répondre à des besoins qui ne sont pas encore couverts par leur nouvelle application.
	\item On pourrait aussi permettre l'envoie de mails depuis le site, pour informer les élèves du déroulement de la gestion de leur mobilité.
\end{itemize}
\smallbreak
L'esthétisme du site pourrait aussi être amélioré, et de nouvelles options d'ergonomie rendraient l'application plus agréable à utiliser.
