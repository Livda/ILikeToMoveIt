\chapter{Avancement du projet}

La version finale de notre projet présente aujourd'hui une application fonctionnelle proposant les fonctionnalités nécessaires pour gérer automatiquement les affectations, les contrats d'études et le suivit des résultats scolaires des étudiants en mobilité.
Nous avons donc réalisé un des principaux objectif, à savoir proposer un application pouvant servir dans le futur. Nous avons utilisé une technologie répandue (Symfony2), ce qui laisse le champs libre à une amélioration du projet dans le futur.
\smallbreak
Si le projet devait se poursuivre et être amélioré, voilà quelques fonctionnalités auxquelles il serait bon de s'intéresser :
\begin{itemize}
	\item Nous n'avons pas implémenté l'ajout de commentaires détaillés sur les universités. L'avis des élèves que nous avons interrogés serait de créer une page comprenant toutes les informations utiles pour chaque destination. Il s'agirait donc de créer une sorte de \textit{Wiki} pour les universités partenaires de l'INSA de Rennes.
	\item Il serait intéressant d'intégrer le service RI, en leur permettant d'avoir accès aux contrats d'étude actualisés (il peut y avoir des changements).
	\item On pourrait aussi permettre l'envoie de mails depuis le site, pour informer les élèves du déroulement de la gestion de leur mobilité.
\end{itemize}
\smallbreak
L'esthétisme du site pourrait aussi être amélioré, et de nouvelles options d'ergonomie rendraient l'application plus agréable à utiliser.

\section{Retours lors du Showroom 27/05}

La présentation de notre projet lors du Showroom au département INFO nous à permis de récolter des retours importants, qui serviront en cas de poursuite du projet.
\bigbreak
\begin{itemize}
	\item Limiter ne nombre de vœux pour éviter les choix irréfléchis, qui pourraient mener à des demandes d'annulation de mobilité,
	\item Générer les contrats d'étude automatiquement (actuellement au format .doc sur l'intranet et existe sous deux formes : Europe et hors-Europe),
	\item Ajouter des informations lors du choix des matières (code de la matière, lien vers le descriptif ...),
	\item Appliquer la signature automatiquement,
	\item Ajouter des filtres pour retrouver les élèves plus facilement, avec plus de détail,
	\item Implémenter un Wiki sur les universités partenaires,
	\item Ajouter une case de validation des notes signifiant qu'elle sont en accord avec le relevé de notes,
	\item Intégrer le CAS de l'INSA de Rennes,
	\item Faire le lien avec Moveon : pour la mise à jour des accords et le dépôt des fichiers.
\end{itemize}