\chapter{Compte-rendu des phases de test}

Devant la complexité pour mettre en place des tests automatisés et représentatifs, et les éventuels changements que l'on pouvait apporter à la structure des données, nous avons opté pour plusieurs phases de tests durant chaque étape du développement de notre produit.

\section{Tests en continu}

Pour garantir l'intégrité de l'application, chaque nouvelle fonctionnalité est testée de façon presque exhaustive. Nous essayons de tester tous les cas qui pourraient poser des problèmes avant de l'ajouter au site. C'est là que la plupart des bugs furent trouvés et corrigés.

\section{Test en situation réelle}
Au terme du premier semestre, en décembre dernier, nous avions développé un premier prototype de l'application. Celui-ci devait pouvoir gérer les vœux des élèves et leur affectation. 
Ce premier test nous a notamment fait découvrir des failles de sécurité qui ont été corrigées par la suite. Nous avons aussi du revoir la gestion des élèves redoublants ou en année de césure.

La campagne d'affectation s'est globalement passé sans problème majeur.

\section{Démonstrations aux responsables RI}
Afin de créer une application pouvant servir à tout l'INSA, nous avons rencontré les responsables RI d'autres départements : EII, SRC, GM et SGM. 
\subsection{Démonstration du prototype}
Nous leur avons ainsi présenté notre prototype durant le mois de février. Ceci nous a permis d'avoir un regard extérieur sur l'application, et des retours importants sur les fonctionnalités à développer. Ces séances ne nous ont pas permis de trouver de bugs, mais furent très importantes dans le développement de l'application.
\subsection{Test de l'application finale}
Nous avons aussi pu recontacter Philippe Mary du département SRC pour lui présenter notre projet fini, durant le mois de Mai. Cette séance nous a permis de découvrir un problème lors de l'import d'universités dans la base de données, ainsi que certaines fonctionnalités utiles aux responsables RI (suppression d'université ou d'étudiant en cas d'erreur à l'ajout, et des options d'ergonomie pour l'ajout d'étudiants). 

\section{Tests utilisateurs}
Différents types d'utilisateurs ont pu tester notre application à différentes étapes du développement.
\subsection{Responsables RI (encadrants)}
Nous avons eu l'aide de nos encadrants Nikolaos Parlavantzas et Christian Raymond pour tester l'application en continu. Leurs retours nous ont servis à enrichir les spécifications et à améliorer l'ergonomie du site. Il nous ont par exemple suggérés d'indiquer par une couleur quel vœux a eu un élève (premier, second, pas de vœux retenu, etc), ou encore de ramener l'affichage à l'étudiant qu'on affecte automatiquement (plutôt que de ramener en haut de la page à l'actualisation).

\subsection{Étudiants}
Plusieurs étudiants de divers départements ont aussi été sollicités pour tester la partie \ogélèves\fg de l'application. Là encore, leurs retours nous ont permis d'améliorer notre application. On nous a par exemple suggérer d'ajouter un lien vers l'accueil dans le bandeau du site, d'afficher l'université à laquelle l'étudiant est affecté en évidence, ou encore de rendre le message d'erreur plus visible quand il fait un vœux qui lui est inaccessible (un quatrième année voulant partir au S8 par exemple).
