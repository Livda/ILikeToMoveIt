\chapter{Introduction}

Ceci est le rapport final concernant notre projet de quatrième année sur la création d'un système de gestion informatisé des mobilités sortantes. Nous y décrirons les différents tests effectués pour garantir la stabilité de l'application, et sa capacité à répondre aux attentes de nos clients, les responsables RI et les élèves. Nous mettrons ensuite en parallèle ce qui a été fait durant le développement avec les rapports de spécifications fonctionnelles et de conception rédigés précédemment. Ce rapport est donc à mettre en parrallèle avec les précédents. Nous évoquerons enfin l'état actuel de notre projet, en spécifiant l'avancement et les éventuelles améliorations à faire, dans le cas ou ce projet devrait être poursuivit.

\smallbreak

Pour rappel, ce projet à des enjeux concrets, et est destiné à devenir une application pérenne pour les étudiants et les professeurs de l'INSA de Rennes. La gestion des mobilité est une tâche complexe, qui demande beaucoup de temps. Nous cherchons donc à aider les responsables RI (Relations Internationales) en leur proposant d'informatiser cette tâche, ce qui leur permettra d'automatiser le processus d'affectation, et d'obtenir une meilleure visibilité des mobilités de leur département. De l'autre côté, nous aideront les élèves en regroupant dans une application, tout ce qu'il faut pour faire leurs choix de destinations, et pour gérer l'aspect pédagogique de leur mobilité (contrat d'étude, relevé de note, etc).

\smallbreak

Rappelons aussi que le déroulement de la mobilité comporte trois phases (simplifiées ici) : 
\begin{itemize}
	\item vœux et affectation des élèves
	\item dépôt et validation des contrats d'étude
	\item dépôt des notes et édition des fiches de jury
\end{itemize}