\chapter{Gestion de fichiers : FileBundle}

Les fonctionnalités présentes dans le FileBundle sont les fonctions permettant la gestion des différents documents. En effet, pour organiser sa mobilité, un étudiant à besoins de plusieurs documents qu'il doit transmettre pour les faire signer puis les récupérer et les redonner etc. Pour faciliter cet échange, plusieurs fonctionnalités sont misent en place.

\section{Dépôt de documents}

Les utilisateurs ont la possibilité de déposer différents documents nécessaires à la mobilité. D'abord le dépôt nécessaire du learning agreement qui sera signer par les administrateurs mais aussi le relevé de notes de l'élève et enfin d'autres documents qui pourraient être demandé suivant les destinations.

\section{Téléchargement de documents}

Les documents déposés sur le site peuvent être téléchargés par l'élève concerné et les administrateurs. Cela permet aux administrateur de récupérer les learning agreement à signer ainsi que le relevé de notes de l'élève. L'élève, lui, peut récupérer son learning agreement signé pour le transmettre à l'école par exemple.

\section{Suppression de documents}

Il est possible à l'élève concerné de supprimer ses documents, si il change de learning agreement, si il dépose un mauvais document,etc. 

\section{Signature électronique}

Pour simplifier la tâche aux administrateur, il est possible de signer les learning agreement numériquement pour ne pas avoir à les imprimer et donc économiser du temps et de l'argent.