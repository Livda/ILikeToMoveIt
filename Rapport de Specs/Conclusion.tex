\chapter*{Conclusion}
\addcontentsline{toc}{chapter}{Conclusion}	

Au terme de cette étape, nous avons défini un premier design de notre application, en expliquant les fonctionnalités offertes aux utilisateurs. 
Les chapitres \ref{chap::etudiants} et \ref{chap::admins} présentent l'interface offerte à chaque type d'utilisateur par notre application. Le chapitre \ref{chap::universites} explique comment les utilisateurs accèderont aux informations sur les universités.
Enfin, le chapitre \ref{chap::archi_logicielle} présente succinctement l'architecture logicielle de notre application.
Nous pouvons aussi présenter une ébauche des vues accessibles, bien que l'apparence même soit sans doute sujette à changement.
\bigbreak
Nous allons désormais travailler à l'implémenter puis tester un premier prototype fonctionnel. Celui-ci répondra à la première étape d'un cycle de départ en mobilité, c'est-à-dire les vœux des étudiants puis leur affectation. L'objectif étant d'avoir à disposition une version incomplète mais fonctionnelle de l'application, pour la tester en situation réelle courant Décembre. 