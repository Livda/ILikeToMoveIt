\section{Tableau des destinations}
Une page de notre application servira à renseigner la liste des destinations accessibles aux étudiants.
Les universités seront regroupées dans un tableau, pour lequel chaque ligne présentera les informations suivantes :
 \begin{itemize}
 	\item le nom de l'université partenaire
 	\item le pays dans laquelle elle se trouve
 	\item les départements pour lesquels cette destination est disponible
 	\item le nombres d'étudiants ayant fait ce vœux, avec un lien vers la liste des élèves en question
 	\item un bouton, à disposition des étudiants, pour ajouter cette destination à leur liste de vœux
 	\item un lien vers la fiche détaillée de l'université
 \end{itemize}
 
 Les admins auront aussi à disposition des boutons pour éditer chaque destination, dont un bouton pour définir si le partenariat est actif ou non. Un partenariat inactif ne sera pas visible pour les étudiants.
 L'édition d'une entrée du tableau les reconduira vers la fiche où se trouvent toutes les informations sur la destination.
 Les admins pourront aussi supprimer une université du tableau s'ils le souhaitent.
 
 Les admins auront aussi moyen d'ajouter des universités au tableau via un bouton. Ils devront alors remplir une fiche descriptive de la nouvelle destination. L'université sera ajoutée au tableau une fois la fiche validée.
 
 Pour facilité le parcours du tableau, une barre de recherche fonctionnant par mot-clef sera disponible au dessus du tableau. Un utilisateur pourra aussi sélectionner des filtres en haut du tableau. Il pourra filtrer par département, pays, ...