\chapter*{Introduction}
\addcontentsline{toc}{chapter}{Introduction}	

Le Système de gestion informatisée des mobilités est une application web permettant aux \ris de l'INSA de Rennes de faciliter leur travail d'affectation et de suivi des étudiants lors de leur mobilité. Ce projet nous a été confié dans le cadre des projets de quatrième année au département Informatique.

\bigbreak

Afin de déterminer les tenants et aboutissants de notre projet, nous avons précédemment rédigé le rapport de pré-étude. Celui-ci contient une explication détaillée des fonctionnalités globales que doit posséder notre projet. Il explique également les technologies utilisées et les raisons qui nous ont poussé à faire ces choix. 

L'application Web sera donc développée en \php, à l'aide du framework \symfony. Elle utilisera le système de base de données \mdb et sera hébergée au Centre de Ressources Informatiques (CRI) sur un serveur \textit{Nginx}. Nous utiliserons enfin le système CAS (Central Authentication Service) du CRI pour nous permettre d'authentifier les utilisateurs.

\bigbreak

Dans la continuité du rapport de pré-étude, nous vous présentons ici l'ensemble des spécificités du cahier des charges. Elles sont liées aux demandes des \ris, et permettent de préciser les fonctionnalités globales. Le projet étant une application web, vous retrouverez principalement des explications sur les vues qui seront développées. Les premières seront les vues auxquelles ont accès les étudiants, puis celle des administrateurs, et enfin les vues liées au universités, qui sont communes à ces deux ensembles.