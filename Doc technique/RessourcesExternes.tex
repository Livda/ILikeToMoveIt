\chapter{Ressources externes}
Le gestionnaire de ressource utilisé pour les Bundle Symfony est composer, via le composer.json du projet pour les dépendances.
Voici les principaux Bundle qui ont été ajouté:
 \newline
\begin{enumerate}

\item Doctrine : ORM avec une documentation assez fournie
 \newline
 \item FOSUser : Gestionnaire d'utilisateur
 \newline	"friendsofsymfony/user-bundle": "2.0.0-alpha3"
	 \newline
\item Fr3D/ldap-Bundle : Sert pour la connexion avec le ldap, il appel zendframework/zen-ldap, qui as été fixé en version 2.6.0 car dans la version suivante il ne gérais pas la multiplicité des comptes ldap et locaux, nous n'avons pas eu le temps de nous pencher sur ce problème
     \newline    "fr3d/ldap-bundle": "2.0.*@dev"
         \newline
\item goodby/csv pour l'import et l'export de .csv
    \newline     "goodby/csv": "dev-master"
 \newline
\end{enumerate}


Nous utilisons aussi deux modules javascript en licence MIT :

 \begin{enumerate}

\item Awesomplete : pour l'autoComplétion, utilisé en version basique mais modifié pour que le séparateur de data ne soit plus "," mais "@"
\newline https://leaverou.github.io/awesomplete/
	\newline
\item TableFilter : filtre pour les tableaux
\newline https://github.com/koalyptus/TableFilter/	
\newline

\end{enumerate}