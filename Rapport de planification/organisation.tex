\chapter{Organisation du projet}

Nous avions plusieurs dates limites définies : un rapport de pré-étude fin octobre, un rapport de spécification mi novembre, ce rapport et une version du site valable pour pouvoir le tester en condition réelle le 18 décembre.

Dans un premier temps nous avons dû choisir les technologies a utiliser. Pour faire cela chaque personne s'est vue attribué une des famille de technologie (serveur, langage, etc) à étudier pour choisir la plus adaptée.
Une fois toutes les technologies sélectionnées et le rapport de pré-étude fini, s'en est suivi une période d'apprentissage des technologies, de définition des spécifications logicielles ainsi que du design de la base de donnée.

Nous avons pour notre projet l'obligation de tester en conditions réelles le site. Pour cela nous avons dû découper le projet en plusieurs modules séparés. Nous avons défini des ordres de priorités pour tout ces modules, certains devant être terminé le 18 décembre impérativement pour permettre le test en condition réelle.
Dés que la base de donnée a été définie, nous avons pu commencer le développement en parallèle de la rédaction du rapport de spécification. Pour facilité le travail à plusieurs nous avons utilisé la plateforme github.
Une fois ces modules répartis entre les membres du projet, chacun développe et teste sa partie indépendamment des autres.
Pour permettre des tests continus un jeu de test réduit a été mis en place et complété au fur et à mesure de l'avancement du projet.
