\chapter{Organisation du projet}

Nous avions plusieurs dates limites définies : un rapport de pré-étude fin octobre, un rapport de spécification mi-novembre, ce présent rapport et une version du site valable pour pouvoir le tester en condition réelle le 18 décembre.

Dans un premier temps nous avons dû choisir les technologies a utiliser. Pour faire cela chaque personne s'est vue attribué une famille de technologie (serveur, langage, etc) à étudier pour choisir la plus adaptée.
Une fois toutes les technologies sélectionnées et le rapport de pré-étude fini, s'en est suivi une période d'apprentissage des technologies, de définition des spécifications logicielles ainsi que du design de la base de donnée.

Notre projet a une particularité : une version fonctionnelle doit être développée pour le 18 décembre. A cette date, le site que nous créons sera mis en ligne et il fonctionnera. De ce fait, nous le testons partiellement à ce moment là en conditions réelles. Afin de répondre à cette contrainte, nous avons découpé le projet en modules. Pour rester cohérent avec l'échéance proche de la mise en ligne, nous avons défini des ordres de priorité en fonction des modules indispensables pour le déploiement.

Une fois la modélisation de la base de données finie, chacun des membres de notre équipe ont pu commencer à developper sur les modules qui leurs avaient été attribués. Afin de faciliter la mise en commun du travail de chacun, nous avons un gestionnaire de version en ligne : \git.

Pour permettre des tests continus, un jeu de test réduit a été mis en place et complété au fur et à mesure de l'avancement du projet. Ainsi, chacun des membres peut l'utiliser et vérifier que ses nouvelles fonctionnalités n'impactent pas sur le bon fonctionnement du site.

En parallèle du travail sur le code du site, nous avons rédigé les différents rapports pour les échéances prévues.

Dans une telle optique de projet, nous opté pour une méthode développement avec des cycles semi-itératifs. Nous travaillons par prototypes fonctionnels pour chacune des phases. Chaque prototype devant permettre une avancée dans notre site et apporter de nouvelles fonctionnalités. Avant chaque déploiement, une vérification de la conformité des attentes de nos encadrants est effectuée.