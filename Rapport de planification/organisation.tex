\chapter{Organisation du projet}

Nous avions plusieurs dates limites définis : un rapport de pré-étude fin octobre, un rapport de spécification mi novembre, ce rapport et une version du site valable pour pouvoir le tester en condition réelle le 18 décembre.

Dans un premier temps nous avons dû choisir les technologies a utilisées. Pour faire cela chaque personne s'est vue attribué une des famille de technologie (serveur, langage, etc) à étudier pour choisir la plus adaptée.
Une fois toutes les technologie selectionnée et le rapport de pré-étude fini, s'en est suivi une période d'apprentissage des technologies, de définition des spécifications logicielles et du design de  la base de donnée.

Nous avons pour notre projet l'obligation de tester en conditions réelle le site. Pour cela nous avons dû découper le projet en plusieurs modules séparés. Nous avons défini des ordres de priorités pour tout ces modules, certains devant être terminé le 18 décembre impérativement pour permettre le test en condition réelle.
Dés que la base de donnée a été défini, nous avons pu commencer le développement en parallèle de la rédaction du rapport de spécification
Une fois ces modules réparties entre les membres du projet, chacun développe et test sa partie indépendamment des autres.
Pour permettre des tests continue un jeu de test réduit a été mis en place au fur et à mesure des besoins.
