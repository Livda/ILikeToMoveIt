\chapter{Super administrateur}

\section{Prélude}

Le super administrateur possède tout les droits d'un administrateur mais n'est pas restreint à un seul département. De ce fait, il possède toutes les capacités d'un administrateur standard. Les possibilités de l'administrateur standard ne seront pas présentés ici, aussi veuillez vous référer au chapitre concernant les administrateurs (Chapitre \ref{admin}) pour plus d'informations sur leurs possibilités. 

\section{Présentation de l'interface}

  \subsection{Accueil}
  \begin{figure}[H]
  	\centering
  	
  	\includegraphics[width=16cm,height=10cm]{Images/Super_Admin/menu_acceuil_super_admin}
  	\caption{Accueil}
  	\label{asa}
  \end{figure}
  
  \begin{enumerate}
  	\item Le super administrateur choisit ici le département de son choix et appuie sur Valider pour effectuer le changement.
  \end{enumerate}
  
    \subsection{Liste des universités partenaires}
    \begin{figure}[H]
    	\centering
    	
    	\includegraphics[width=16cm,height=12cm]{Images/Super_Admin/liste_univ_super_admin}
    	\caption{Liste des universités partenaires}
    	\label{lusa}
    \end{figure}
 	
 	\begin{enumerate}
 		\item Le super administrateur peut supprimer définitivement une université de la base de donnée. Elle sera alors supprimé en cascade pour tous les départements, ainsi que les vœux pour cette université.
 	\end{enumerate}
 
  \subsection{Ajout d'élèves ou d'administrateurs}
  \begin{figure}[H]
  	\centering
  	
  	\includegraphics[width=16cm,height=12cm]{Images/Super_Admin/ajout_gens_super_admin}
  	\caption{Ajout d'élèves ou d'administrateurs}
  	\label{agsa}
  \end{figure}
    
    \begin{enumerate}
    	\item En plus de l'ajout d'étudiants, le super administrateur peut ajouter un administrateur en utilisant l'adresse mail (INSA de rennes). L'administrateur sera ajouté pour le département courant du super administrateur.
    \end{enumerate}