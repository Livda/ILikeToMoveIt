\chapter{Administrateur}
\label{admin}

\section{Presentation de l'interface}
Ci-dessous vous pouvez trouver une présentation de l'interface proposée aux administrateur. Dans un soucis de clarté et pour évité de répéter plusieurs fois la même information, en haut de chaques tableaux présents sur le site vous pouvez noter la présence de cases blanches permettant de filtrer les tableaux avec les informations de votre choix puis appliquer ces filtre en appuyant sur la touche entrée de votre clavier.

Pour supprimer tous les filtres, vous pouvez appuyer sur le symbole marqué d'une croix rouge en haut à droite des différents tableaux.

 
\subsection{Accueil}
\label{aa}
Cette sur cette page que l'administrateur arrive à sa connexion. Depuis ce hub, il est possible d'accéder à la grande majorité des pages importante du site.
 \begin{figure}[H]
 	\centering
 	
 	\includegraphics[width=16cm,height=13cm]{Images/Admin/menu_acceuil_admin.png}
 	\caption{Accueil administrateur}
 	
 \end{figure}
 
\begin{enumerate}
\item accès à la liste des étudiant (cf \ref{lua}),
\item accès à la modification des jetons (cf \ref{gua}),
\item accès à la page d'ajout d'universités (cf \ref{pua}),
\item accès à la liste des vœux (cf \ref{lv}),
\item accès à la page permettant d'effectuer l'affectation des élèves (cf \ref{ae}),
\item accès à la page d'ajout des étudiants (cf \ref{aet}),
\item accès à la page de génération de fiches de jury (cf \ref{fj}),
\item ce bouton génère la liste des mails des élèves présent dans le tableau après application des filtres,
\item ce bouton génère un fichier contenant les informations sur les élèves ayant fait des vœux dans l'année courante. Ce fichier CSV peut par la suite être utilisé grâce à excel par exemple,
\item au bout de 5 ans, un élève est considéré comme inactif et n'apparait donc plus dans le tableau. Il est possible de changer cela en cliquant sur ce bouton,
\item ce bouton permet d'accéder à la fiche d'un élève en mode administrateur (cf \ref{ev} , \ref{en} , \ref{ef}),
\item ce bouton permet d'accéder à la fiche d'une université (\ref{fu}).
\end{enumerate}

 
 \subsection{Liste des universités partenaires}
 \label{lua}
 Cette page contient la liste de toutes les universités partenaire de votre département pour lesquelles les élèves peuvent faire des vœux.
 \begin{figure}[H]
 	\centering
 	
 	\includegraphics[width=14cm,height=10cm]{Images/Admin/liste_univ_admin.png}
 	\caption{Liste des universités partenaires}
 	
 \end{figure}
 \begin{enumerate}
 \item ce bouton permet d'accéder à la fiche d'une université (cf \ref{fu}).
 \end{enumerate}
 
 \bigbreak\bigbreak\bigbreak
  \subsection{Modifier/Ajouter des jetons/Ajouter des Universités}
  \label{gua}
  Sur cette page il est possible de gérer les jetons disponibles aux élèves de votre département.
  
  \begin{figure}[H]
  	\centering
  	
  	\includegraphics[width=16cm,height=9cm]{Images/Admin/gestion_univ_admin.png}
  	\caption{Modifier/Ajouter des jetons/Ajouter des Universités}
  	
  \end{figure}
   \begin{enumerate}
   	\item cette ligne du tableau permet d'ajouter des nouveaux jetons pour les élèves (cf \ref{cj} pour connaître la procédure à suivre). \att ajouter des jetons à une université qui n'existe pas aura pour effet de l'ajouter dans la base de données,
   	\item ce bouton permet d'accéder à la fiche d'une université (\ref{fu}), 
   	\item ce groupe de champs permet de modifier le nombre de jeton pour une université ainsi que les types de mobilités disponible (cf \ref{mj}),
   	\item ce bouton permet de supprimer les jetons correspondant à la ligne sélectionnée. \att, ce bouton n'entraine pas la suppression totale de l'université mais seulement de ses jetons.
   \end{enumerate}

\newpage
  \subsection{Ajout de plusieurs universités}
  Cette page permet l'ajout de nouvelle universités. l'ajout d'universités se fait par le biais d'un fichier CSV dont le format est précisé sur la page dans la case "Format". Pour plus d'information consultez la section \ref{au}. \att, il est possible d'ajouter une unique université sur la page de gestion des jetons en ajoutant des jetons à une université qui n'existe pas encore.
  \begin{figure}[H]
  	\centering
  	\includegraphics[width=16cm,height=9cm]{Images/Admin/ajout_plusieux_univ_admin.png}
  	\caption{Ajouter plusieurs universités}
  	
  \end{figure}
  \begin{enumerate}
  	\item ce bouton permet d'ouvrir votre explorateur de dossier pour sélectionner le fichier CSV contenant les universités que vous souhaitez ajouter,
  	\item ce menu déroulant permet de sélectionner les mobilités par défaut,
  	\item ce menu déroulant permet de choisir le séparateur utilisé dans votre fichier CSV.\att à bien choisir le CSV correspondant à votre fichier CSV,
  	\item ce bouton permet de valider vos options et valider lancer l'import de vos universités via le CSV.
  \end{enumerate}
  
    \subsection{Fiches de jurys}
    \label{fj}
    \begin{figure}[H]
    	\centering
    	\includegraphics[width=16cm,height=9cm]{Images/Admin/fiche_jury_admin.png}
    	\caption{Fiches de jurys}
    	
    \end{figure}
    \begin{enumerate}
    \item cette ligne permet la génération des fiches de jury. Référez vous à la section \ref{fju},
    \item en cliquant sur ce bouton vous accéder à la page étudiant en mode administrateur (cf \ref{ev}, \ref{en}, \ref{ef}).
    \end{enumerate}
    
    
      
 \subsection{Fiches universités}
 \label{fu}
 Sur cette page sont présente les informations relatives à une universités. On trouve d'abords la liste des départs possibles pour l'universités indépendamment du département. Ensuite le second tableau montre tous les étudiants de votre département ayant fait un vœux pour cette université puis un tableau contenant la liste des tous les élèves déjà partit dans cette université, indépendamment de leur département.
   \begin{figure}[H]
      	\centering
       \includegraphics[width=16cm,height=10cm]{Images/Admin/fiche_univ_admin.png}
       \caption{Fiches université}
       
  \end{figure}
  
   \subsection{Liste des vœux}
   \label{lv}
   \begin{figure}[H]
   	\centering
   	\includegraphics[width=18cm,height=10cm]{Images/Admin/liste_voeux_admin.png}
   	\caption{Liste des vœux} 	
   \end{figure}
   \begin{enumerate}
   	\item
   	\item
   	\item
   	\item
   	\item
	\end{enumerate}
   
    \subsection{Affectation des élèves}
    \label{ae}
    \begin{figure}[H]
    	\centering
    	\includegraphics[width=16cm,height=14cm]{Images/Admin/moulinette_admin.png}
    	\caption{Affectation des élèves}
    	
    \end{figure}
    
         \subsection{Page étudiant : Vœux}
         \label{ev}
         \begin{figure}[H]
         	\centering
         	\includegraphics[width=16cm,height=12cm]{Images/Admin/page_etud_admin.png}
         	\caption{Page étudiant : Vœux}
         	
         \end{figure}
         
    
     \subsection{Page étudiant : Contrat d'étude et notes}
     \label{en}
     \begin{figure}[H]
     	\centering
     	\includegraphics[width=16cm,height=12cm]{Images/Admin/note_admin.png}
     	\caption{Page étudiant : Contrat d'étude et notes}
     	
     \end{figure}
     
     
          \subsection{Page étudiant : Dépôt de fichier}
          \label{ef}
          \begin{figure}[H]
          	\centering
          	\includegraphics[width=16cm,height=12cm]{Images/Admin/ajout_fichier_admin.png}
          	\caption{Page étudiant : Dépôt de fichier}
          	
          \end{figure}
          
          \subsection{Ajout d'étudiants}
          \label{aet}
          \begin{figure}[H]
          	\centering
          	\includegraphics[width=14cm,height=10cm]{Images/Admin/ajout_etud_admin.png}
          	\caption{Ajout d'étudiants}
          	
          \end{figure}
        


\section{Gestion des universités}

\subsection{Voir la fiche résumé des universités}

Depuis la page d'accueil \ref{aa} cliquez sur le lien "Liste des universités partenaire". Une fois sur cette page, il vous suffit de cliquer sur le nom de l'université de votre choix pour arriver sur sa page personnelle.
 
\subsection{Modifier les jetons existant d'une université}
\label{mj}
Depuis la page d'accueil \ref{aa} cliquez sur le lien "Modifier/Ajouter des jetons/universités". Vous arrivez sur une page contenant une table avec la liste des jetons associés à une universités pour les semestres cochés. A droite de chaque université, est présente une case "Modifier" dans laquelle vous pouvez modifier la liste des départs disponible à cette destination ainsi que le nombre de place disponible.

\smallbreak

Par exemple, pour une université donnée, les cases S8 et S9 cochés la valeur de 2 pour le nombre de jetons signifie qu'il n'y a que deux places disponibles pour les départs S8 et S9 et non pas 2 places pour S8 et deux places pour S9.

\smallbreak

PS: mettre une valeur de -1 comme nombre de place signifie qu'il y a un nombre illimité de place disponibles.

\subsection{Créer des nouveaux jetons pour une universités}
\label{cj}

Depuis la page de modification des universités (cf paragraphe précédent pour atteindre cette page), si vous souhaitez mettre un nombre de jeton différents pour deux mobilités différentes dans une même université, vous devez créer une nouvelle université ayant le même nom que celle déjà existante puis modifier de manière indépendantes les deux universités.

\smallbreak

Cette méthode doit être utilisée si on souhaite donner un nombre de départs indépendants à deux types de mobilités différentes. Il suffit d'ajouter un nouveau groupe de jeton avec le même nom d'université, et de cocher différentes cases.


\subsection{Ajouter plusieurs universités} 
\label{au}

Depuis la page d'accueil \ref{aa} cliquez sur le lien "Ajoutez plusieurs universités d'un coup". Sur cette page, vous devez uploader un fichier CSV ayant le format décrit dans la case "Format". \att le fichier CSV doit être enregistré au format UTF-8 pour que l'import fonctionne correctement). Choississez le séparateur correspondant à votre fichier CSV puis cliquer sur valider.

\section{Gestion manuelle des voeux des étudiants}

Depuis la page d'accueil \ref{aa} cliquez sur le lien "Gérer les voeux". Vous arrivez sur une page contenant un tableau résumant par élève la liste de ses 3 premiers voeux. Cliquez ensuite sur "Acceder" pour arriver sur la page élève et pouvoir modifier la liste de ses voeux (ajouter, supprimer) et l'ordre de ses voeux à la place de l'élève (cf section choix des destination du chapitre élève).

\section{Fin de la phase de vœux} 

Depuis la page d'accueil \ref{aa} cliquez sur le lien "Gérer les affectations". Cliquez sur le bouton "Bloquer la possibilité de modifier les veux afin que les élèves ne puissent plus modifier ou ajouter de nouveaux vœux.

\section{Affectation des vœux}

Sur la page de gestion des affectation (cf paragraphe précédent"), tout d'abords

\section{Liste de mails}

Depuis la page d'accueil \ref{aa}, avec les filtres selectionner la liste des étudiants qui vous intéressent puis cliquer sur "Afficher les mails des élèves visibles". Il vous suffit ensuite de copier la liste obtenue dans le champs destinataire de votre logiciel de mail.

\section{Gestion des contrat d'étude}

\subsection{Accepter contrat d'étude}
Depuis la page d'accueil, cliquez sur le bouton "acceder" de la ligne correspondant à l'élève souhaité. Un fois sur sa page, si l'élève vous a soumis un contrat d'études, vous pouvez soit le valider en cliquant sur la touche "valider", ou alors le refuser en écrivant si vous le souhaiter une raison à votre refus.

\subsection{Rentrer manuellement un contrat d'étude}
Depuis la page d'un élève, vous pouvez ajoutez manuellement un learning agreement et le valider directement.

\section{Dépôt de documents}

Depuis la page d'un étudiant, dans la section "Upload de documents, vous pouvez ajouter des fichiers pour l'élève en question. Pour cela, entrez le nom souhaité pour le document puis cliquer sur "choisissez un fichier" et sélectionnez le fichier. Cliquez enfin sur "envoyer" 

\section{Gérer les notes des élèves}

\section{Générer les fiches de jury}
\label{fju}

