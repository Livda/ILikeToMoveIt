\chapter*{Conclusion}
\addcontentsline{toc}{chapter}{Conclusion}	

Ce projet a pour but d'éconnomiser du temps et des photocopies pour toute l'INSA de Rennes, grâce à une application permettant la gestion informatisée des mobilités à l'étranger.

\bigbreak

La répartition des ressources semble assez légère pour la construction de l'application car nous devrons réserver beaucoup de temps aux tests. En effet, nous avons la responsabilité du suivi des mobilités de toute une année au département INFO, et nous devons être sûr du bon fonctionnement de l'application.

\bigbreak

La planification de ce projet est donc assez originale puisqu'elle ne suis pas de méthode de développement classique (V,agile...). À la place de cela, nous utiliserons une méthode intermediaire, par itération respectant les trois étapes de la gestion des mobilités. Au terme de chaque étape, une nouvelle version fonctionnelle du projet permettra aux étudiants et aux responsables RI de gérer plus efficacement les mobilités. 
