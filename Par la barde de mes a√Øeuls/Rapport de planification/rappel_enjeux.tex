\chapter{Enjeux du projet}


\textbf{Rappel des exigences}
\bigbreak

L'application va permettre d'aider les \ris à simplifier la gestion des mobilités des étudiants. 

\bigbreak

Les étudiants et les administrateurs devront pouvoir s'authentifier et ainsi avoir accès à une interface qui leur est spécifique. Le LDAP (annuaire électronique) du CRI (Centre de Ressources Informatique) sera utilisé à cet effet. Le but de l'application est d'attribuer à chaque étudiant qui le souhaite, une école pour y faire sa mobilité. L'affectation et le suivi des mobilités sont séparés en trois phases, ou étapes. 


\bigbreak

La première correspond, pour les étudiants (ou élèves dans le rapport), à la recherche de l'université de destination et au choix de leurs \voe. Ils doivent donc pouvoir accéder à la liste des écoles. Une liste de commentaires des élèves déjà partis vers ces destinations doit également être présente.

Les \ris peuvent gérer les étudiants, les écoles partenaires, les \voe des étudiants et enfin leur affectation. Les affectations peuvent être effectuées de façon manuelle, ou de façon automatique à l'aide d'un algorithme qui pourra être choisi. Lorsque chaque étudiant a une école qui lui est affectée, la deuxième phase commence.
\bigbreak

Certains documents sont nécessaires pour la mobilité. Ils devront être accessibles en téléchargement, et une génération de façon automatique peut être envisagée. Les \ris et membres du SRI (Service Relations Internationales) doivent pouvoir récupérer ces fichiers. Ils disposeront donc de certaines vues en commun avec les \ris, mais ne pourront pas valider les affectations des étudiants. Enfin, les documents devront être signés de façon électronique.

\bigbreak

La troisième et dernière phase correspond au suivi des mobilités après le départ des étudiants. Le contrat de mobilité doit pouvoir être modifié par l'élève, et les \ris doivent en être notifiés. Ils pourront ensuite valider le nouveau contrat. A la fin de la mobilité, l'étudiant pourra entrer ses notes sur le site pour le jury de fin d'année. Il pourra également laisser un commentaire sur l'université pour aiguiller les étudiants intéressés. La demande de génération de fiches de jury pourra être demandée par les \ris.


\bigbreak

Afin d'avoir une véritable architecture de travail, nous nous sommes tournés vers le CRI. Ce dernier nous a fournis un serveur web avec des accès pour les membres du groupe. Nous avons donc installé un serveur \textit{Ngninx} et une base de données \mdb, pour finalement obtenir une base de développement et de test pour l'application.